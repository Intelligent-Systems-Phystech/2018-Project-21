%%%%%%%%%%%%%%%%%%%%%%%%%%%%%%%%%%%%%%%%%
% Beamer Presentation
% LaTeX Template
% Version 1.0 (10/11/12)
%
% This template has been downloaded from:
% http://www.LaTeXTemplates.com
%
% License:
% CC BY-NC-SA 3.0 (http://creativecommons.org/licenses/by-nc-sa/3.0/)
%
%%%%%%%%%%%%%%%%%%%%%%%%%%%%%%%%%%%%%%%%%

%----------------------------------------------------------------------------------------
%	PACKAGES AND THEMES
%----------------------------------------------------------------------------------------

\documentclass[12pt]{beamer}
\mode<presentation> {

% The Beamer class comes with a number of default slide themes
% which change the colors and layouts of slides. Below this is a list
% of all the themes, uncomment each in turn to see what they look like.

%\usetheme{default}
%\usetheme{AnnArbor}
%\usetheme{Antibes}
%\usetheme{Bergen}
%\usetheme{Berkeley}
%\usetheme{Berlin}
%\usetheme{Boadilla}
%\usetheme{CambridgeUS}
%\usetheme{Copenhagen}
%\usetheme{Darmstadt}
%\usetheme{Dresden}
%\usetheme{Frankfurt}
%\usetheme{Goettingen}
%\usetheme{Hannover}
%\usetheme{Ilmenau}
%\usetheme{JuanLesPins}
%\usetheme{Luebeck}
\usetheme{Madrid}
%\usetheme{Malmoe}
%\usetheme{Marburg}
%\usetheme{Montpellier}
%\usetheme{PaloAlto}
%\usetheme{Pittsburgh}
%\usetheme{Rochester}
%\usetheme{Singapore}
%\usetheme{Szeged}
%\usetheme{Warsaw}

% As well as themes, the Beamer class has a number of color themes
% for any slide theme. Uncomment each of these in turn to see how it
% changes the colors of your current slide theme.

%\usecolortheme{albatross}
%\usecolortheme{beaver}
%\usecolortheme{beetle}
%\usecolortheme{crane}
%\usecolortheme{dolphin}
%\usecolortheme{dove}
%\usecolortheme{fly}
%\usecolortheme{lily}
%\usecolortheme{orchid}
%\usecolortheme{rose}
%\usecolortheme{seagull}
%\usecolortheme{seahorse}
%\usecolortheme{whale}
%\usecolortheme{wolverine}

%\setbeamertemplate{footline} % To remove the footer line in all slides uncomment this line
%\setbeamertemplate{footline}[page number] % To replace the footer line in all slides with a simple slide count uncomment this line

%\setbeamertemplate{navigation symbols}{} % To remove the navigation symbols from the bottom of all slides uncomment this line
}
\usepackage[utf8]{inputenc}
\usepackage{adjustbox}
\usepackage[T2A]{fontenc}
\usepackage{amssymb}
\usepackage{amsmath}
\usepackage{mathrsfs}
\usepackage{euscript}
\usepackage{upgreek}
\usepackage[english,russian]{babel}
\usepackage{array}
\usepackage{theorem}
\usepackage[all]{xy}
\usepackage{graphicx}
\usepackage{subfig}
\usepackage{epstopdf}   
\usepackage{tikz}       
\usepackage{pgfplots}   
\usepackage{color}
\usepackage{ifthen}
\usepackage{url}
\usepackage{makeidx}
\usepackage{pb-diagram}
\usepackage{balance}
\usepackage{multirow} 
\usepackage{bibentry}
\usepackage{graphicx} % Allows including images
\usepackage{booktabs}
\usepackage{graphics}
\usepackage{cmap}
\usepackage{amsthm}
\usepackage[linesnumbered,ruled,vlined]{algorithm2e}
\usepackage[absolute]{textpos}
\usepackage{fleqn,psfrag,wrapfig,tikz}
\usepackage{algpseudocode}
 
%----------------------------------------------------------------------------------------
%	TITLE PAGE
%----------------------------------------------------------------------------------------

\title[Тензорные методы]{Методы выпуклой оптимизации высокого порядка} % The short title appears at the bottom of every slide, the full title is only on the title page

\author{Даниил Селиханович} % Your name
\institute[МФТИ] % Your institution as it will appear on the bottom of every slide, may be shorthand to save space
{
МФТИ\\ 
ИППИ им. Харкевича РАН \\ % Your institution for the title page
\medskip
\textit{selihanovich.do@phystech.edu} % Your email address
}
\date{\today} % Date, can be changed to a custom date

\begin{document}

\begin{frame}
\titlepage % Print the title page as the first slide
\end{frame}

\begin{frame}
\frametitle{План доклада} % Table of contents slide, comment this block out to remove it
\tableofcontents % Throughout your presentation, if you choose to use \section{} and \subsection{} commands, these will automatically be printed on this slide as an overview of your presentation
\end{frame}

%----------------------------------------------------------------------------------------
%	PRESENTATION SLIDES
%----------------------------------------------------------------------------------------

%------------------------------------------------

\section{Методы высоких порядков выпуклой оптимизации}
\begin{frame}
\frametitle{Тензорный шаг Нестерова}
В работе \cite{nesterov2018implementable}, вышедшей в марте 2018 года, рассматривается следующий итерационный процесс для решения задачи выпуклой оптимизации $\min\limits_{x \in \mathbb{E}} f(x)$:
$$
x_{t + 1} = T_{p, M}(x_t), t \geq 0,
$$
где функция $f(x) \in \mathcal{F}_p$ - класс выпуклых и $p$-раз гладких функций, а $T_{p, M}(x_t) \in \arg\min_{y \in \mathbb{E}}\Omega_{x, p, M}(y)$, 
$$
\Omega_{x, p, M}(y) = f(x) +  \sum\limits_{i = 1}^p \frac{1}{i!}D^if(x)[y - x]^i + \frac{M}{(p - 1)!}\frac{||y - x||^{p + 1}}{p + 1}
$$
при $M \geq L_p$, $L_p$ - константа Липшица для $p$-й производной функции $f$.
\end{frame}
\begin{frame}
\frametitle{Случай задачи с целевой функцией logistic-loss}
Рассматривается задача бинарной классификацией с логистической функцией потерь:
$$
loss(w, y, X) = \sum\limits_{i = 1}^m \ln\Bigl(1 + \exp(-y_i\langle X_i, w \rangle)\Bigr) \to \min_{w \in \mathbb{R}^n},
$$
где $y_i \in \{1, -1\}$ - классы объектов, $m$ - число объектов, $X$ - матрица объектов-признаков с константым признаком 1, $n$ - размерность вектора в решающем правиле линейного классификатора:
$$
y_{predict}(w, x) = sign \langle w, x\rangle.
$$
\end{frame}
\begin{frame}
\frametitle{Оценка констант Липшица для logistic-loss}
Определение константы Липшица для $p$-й производной функции $f$:
$$
||D^p f(x) - D^p f(y)|| \leq L_p||x - y||, \; x,y \in \mathrm{dom} \, f, p \geq 1.
$$
Для $p = 1 \to D^1 f(x) = \nabla f(x)$, для $p = 2 \to D^2 f(x) = \nabla^2 f(x)$.
Можно показать, что в случае евклидовой нормы для такой функции выполнено:
$$
L_1 \leq \frac{1}{4}\sum\limits_{i = 1}^m ||X_i||_2^2 = M_1;
$$
$$
L_2 \leq \frac{1}{6\sqrt{3}}\sum\limits_{i = 1}^m ||X_i||_2^3 \leq \frac{1}{10}\sum\limits_{i = 1}^m ||X_i||_2^3 = M_2. 
$$
\end{frame}
\section{Применение для задачи классификации}
\begin{frame}
\frametitle{Цель работы}
\begin{block}{Постановка задачи}
\begin{itemize}
	\item Применить тензорный метод Нестерова 2-го порядка в задаче бинарной классификации;
	\item Рассмотреть разные способы проведения тензорного шага;
	\item Сравнить результат работы с быстрым градиентным спуском;
	\item Сравнить все алгоритмы с реализацией логистической регрессии в sklearn.
\end{itemize}
\end{block}
\end{frame}
\section{Описание данных}
\begin{frame}
\frametitle{Данные и их описание}
Данные Covertype Data Set представляют собой задачу прогнозирования типа лесного покрова в штате Колорадо по картографическим признакам. Задача мультиклассовой классификации была предобработана в задачу бинарной.
\newline
\begin{itemize}
	\item Всего 581012 объектов и 54 признака.
	\item Из 54 признаков 10 количественные и 44 бинарные.
	\item Количественные признаки нормированы на $[0, 1]$.
	\item В обучение вошли 20000 объектов, в тест - 30000.
	\item Распределение классов: 1-ый класс в train - 13978, в test - 7229, 2-й класс в train - 6022, в test = 22771.
\end{itemize}
\end{frame}

\section{Результаты}
\begin{frame}
\frametitle{График logistic-loss от числа итераций}
Видно, что алгоритмы сопряжённых градиентов и LBFGS в тензорном шаге Нестерова дают очень похожий результат.
\begin{figure}
\includegraphics[width=0.8\linewidth]{all_f_iter.pdf}
\end{figure}
\end{frame}

\begin{frame}
\frametitle{График нормы градиента от числа итераций}
Методы высоких порядков работают не намного лучше быстрого градиентного спуска.
\begin{figure}
	\includegraphics[width=0.8\linewidth]{all_gradf_iter.pdf}
\end{figure}
\end{frame}

\begin{frame}
\frametitle{Точность на train}
Через большое число итераций все методы дают практически одинаковую точность на train.
\begin{figure}
	\includegraphics[width=0.8\linewidth]{accuracy_all_train.pdf}
\end{figure}
\end{frame}

\begin{frame}
\frametitle{Точность на test}
FGM и метод 2-го порядка работают хуже на тестовых данных, чем реализованный в библиотеке LIBLINEAR.
\begin{figure}
	\includegraphics[width=0.8\linewidth]{accuracy_all_test.pdf}
\end{figure}
\end{frame}

\section{Выводы}
\begin{frame}
\frametitle{Выводы}
\begin{itemize}
	\item Методы высоких порядков могут использоваться в задачах ML.
	\item С повышением порядка метода увеличиваются вычислительные расходы по времени и памяти.
	\item Константы Липшица стоит подбирать адаптивно.
	\item Можно выбирать разные алгоритмы для выполнения тензорного шага.
\end{itemize}
\end{frame}



\begin{frame}
\frametitle{Литература}
\bibliographystyle{amsalpha}
\begin{thebibliography}{0}
	
	\bibitem[1]{nesterov2018implementable}
	\emph{Nesterov Y.} Implementable tensor methods in unconstrained convex optimization. – Université catholique de Louvain, Center for Operations Research and Econometrics (CORE), 2018. – №. 2018005.
	
	\bibitem[2]{rodomanov}
	\emph{Родоманов А. О., Кропотов Д. А., Ветров Д. П.} Анализ быстрого градиентного метода Нестерова для задач машинного обучения с $L_1$-регуляризацией.
	
	\bibitem[3]{gasnikov2018grad}
	Современные численные методы оптимизации. Метод универсального градиентного спуска: учебное пособие / А. В. Гасников. – М. : МФТИ, 2018. –	181с. –	Изд. 2-е, доп.
	
	\bibitem[4]{tensormethods}
	\emph{Гасников А. В. и др.} Обоснование гипотезы
	об оптимальных оценках скорости сходимости численных методов выпуклой оптимизации высоких порядков.
	
\end{thebibliography}
\end{frame}

\begin{frame}
\frametitle{Благодарность}
Выражаю благодарность Е. А. Воронцовой и А. В. Гасникову за руководство проектом и обсуждения результатов!
\end{frame}

\begin{frame}
\Huge{\centerline{Спасибо за внимание! Вопросы?}}
\end{frame}

\end{document}